%-------------------------------------------------------------------------------
% File: spark.tex
%       
%
% Author: Marco Pinna
%         Created on 14/06/2022
%-------------------------------------------------------------------------------
\chapter{Spark}\label{ch:spark}
The second version was implemented in Python, using the Spark framework.\\
The directory structure of the project is the following:\\
\dirtree{%
.1 hadoop.
.2 spark.
.3 BloomFilter.py.
.3 spark\_compute\_params.py.
.3 spark\_bloomfilter.py.
.3 spark\_FPR\_test.py.
.3 util.py.
}
\hfill \break
The files organization is quite similar to the one used in Hadoop, with the same classes used for the same purposes.\\
The hash function family is again MurmurHash, an implementation of which can be found in the \texttt{mmh3} Python module.\\
There was no need for a Python equivalent of the Java \texttt{NLineInputFormat} class: the \texttt{SparkContext} class available in the \texttt{pyspark} Python module already has partition size handling built-in within the \texttt{textFile()} method, which simply takes the number of partitions as additional argument when instantiating the first resilient distributed dataset (\textit{RDD}).

%TODO add something about the broadcasts?

\subsection*{Usage}
The three \texttt{spark\_*} files can be run with the following commands:\\

\begin{lstlisting}[basicstyle=\ttfamily\footnotesize, breaklines=true, backgroundcolor = \color{lightgray}]
spark-submit spark_compute_params.py yarn hadoop-namenode 9820 title.ratings.tsv output 0.01
\end{lstlisting}
\hfill \break
\texttt{yarn} is the \textit{master} argument that selects on which FS the script has to be run, \texttt{hadoop-\\namenode} is the hostname, \texttt{9820} is the port, \texttt{title.ratings.tsv} is the name of the output file and \texttt{0.01} is the desired value for the FPR \textit{p}.
\vspace{5pt}
\hrule
\vspace{6pt}
\begin{lstlisting}[basicstyle=\ttfamily\footnotesize,breaklines=true, backgroundcolor = \color{lightgray}]
spark-submit --archives pyspark_venv.tar.gz#environment spark_bloomfilter.py yarn hadoop-namenode 9820 title.ratings.tsv output 8 24404 63482 171928 420900 990060 2117062 3578304 3406703 1092505 156103 7 0.01 WithBloomFilters
\end{lstlisting}
\hfill \break
\texttt{title.ratings.tsv} is the input file, \texttt{output} is the name of the output file, \texttt{8} is the number of partition the dataset has to be split into, the ten numbers that follow are the values for \textit{m} for each filter, \texttt{7} is the value for \textit{k} and \texttt{WithBloomFilters} is a flag that toggles which of the two implementations is to be run.
\vspace{5pt}
\hrule
\vspace{6pt}
\begin{lstlisting}[basicstyle=\ttfamily\footnotesize, breaklines=true, backgroundcolor = \color{lightgray}]
spark-submit --archives pyspark_venv.tar.gz#environment spark_FPR_test.py yarn hadoop-namenode 9820 title.ratings.tsv testOutput output/part-00000
\end{lstlisting}
\hfill \break
\texttt{yarn} is the \textit{master } argument that selects on which FS the script has to be run, \texttt{hadoop-namenode} is the hostname, \texttt{9820} is the port, \texttt{title.ratings.tsv} is the name of the output file, \texttt{testOutput} is the name of the output file, \texttt{output/part-r-00000} is the file from which the saved Bloom filters have to be read.