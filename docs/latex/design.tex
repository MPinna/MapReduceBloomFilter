%-------------------------------------------------------------------------------
% File: design.tex
%       
%
% Author: Marco Pinna
%         Created on 14/06/2022
%-------------------------------------------------------------------------------
\chapter{Overview and design choices}\label{ch:design}
In this chapter we firstly present a brief overview of what Bloom filters are (partly taken from the project specifications); then we show the general algorithm, with two possible implementation along with their respective pseudo-code, together with design choices and hypotheses that have been made during the design process.\\
Finally, some considerations about the dataset to be used and the use cases of the Bloom filter are made.\\

\section{Bloom filters}
A \textit{Bloom filter} is a space-efficient probabilistic data structure that is used for membership testing.\\
A Bloom filter consists of a bit-vector with \textit{m} elements and \textit{k} hash functions to map \textit{n} keys to the \textit{m} elements of the bit-vector.\\
The two possible operations that can be done on a Bloom filter are \texttt{add} and \texttt{test}.\\
Given a key \textit{$id_{i}$}, every hash function \textit{$h_{1}, ..., h_{k}$} computes the corresponding output positions and sets the corresponding bit in that position to 1.\\
The space efficiency of Bloom filters comes at the cost of having a non-zero probability of false positives. The false positive rate of a Bloom filter is denoted by \textit{p}.\\\
Therefore the two possible outcomes of the \texttt{test} function of the Bloom filter are ``Possibly in set" or ``Definitely not in set".\\

\section{Algorithm design}
The general idea of this MapReduce implementation is


%TODO inserire figure